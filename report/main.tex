\documentclass{article}
%%%%%%%%%%%%%%%%%%%%%%%%%%%%%%%%%%%%%%%%%
% Lachaise Assignment
% Structure Specification File
% Version 1.0 (26/6/2018)
%
% This template originates from:
% http://www.LaTeXTemplates.com
%
% Authors:
% Marion Lachaise & François Févotte
% Vel (vel@LaTeXTemplates.com)
%
% License:
% CC BY-NC-SA 3.0 (http://creativecommons.org/licenses/by-nc-sa/3.0/)
% 
%%%%%%%%%%%%%%%%%%%%%%%%%%%%%%%%%%%%%%%%%

%----------------------------------------------------------------------------------------
%	PACKAGES AND OTHER DOCUMENT CONFIGURATIONS
%----------------------------------------------------------------------------------------

\usepackage{amsmath,amsfonts,stmaryrd,amssymb} % Math packages

\usepackage{enumerate} % Custom item numbers for enumerations

\usepackage[ruled]{algorithm2e} % Algorithms

\usepackage[framemethod=tikz]{mdframed} % Allows defining custom boxed/framed environments

\usepackage{listings} % File listings, with syntax highlighting
\lstset{
	basicstyle=\ttfamily, % Typeset listings in monospace font
}

%----------------------------------------------------------------------------------------
%	DOCUMENT MARGINS
%----------------------------------------------------------------------------------------

\usepackage{geometry} % Required for adjusting page dimensions and margins

\geometry{
	paper=a4paper, % Paper size, change to letterpaper for US letter size
	top=2.5cm, % Top margin
	bottom=3cm, % Bottom margin
	left=2.5cm, % Left margin
	right=2.5cm, % Right margin
	headheight=14pt, % Header height
	footskip=1.5cm, % Space from the bottom margin to the baseline of the footer
	headsep=1.2cm, % Space from the top margin to the baseline of the header
	%showframe, % Uncomment to show how the type block is set on the page
}

%----------------------------------------------------------------------------------------
%	FONTS
%----------------------------------------------------------------------------------------

\usepackage[utf8]{inputenc} % Required for inputting international characters
\usepackage[T1]{fontenc} % Output font encoding for international characters

\usepackage{XCharter} % Use the XCharter fonts

%----------------------------------------------------------------------------------------
%	COMMAND LINE ENVIRONMENT
%----------------------------------------------------------------------------------------

% Usage:
% \begin{commandline}
%	\begin{verbatim}
%		$ ls
%		
%		Applications	Desktop	...
%	\end{verbatim}
% \end{commandline}

\mdfdefinestyle{commandline}{
	leftmargin=10pt,
	rightmargin=10pt,
	innerleftmargin=15pt,
	middlelinecolor=black!50!white,
	middlelinewidth=2pt,
	frametitlerule=false,
	backgroundcolor=black!5!white,
	frametitle={Command Line},
	frametitlefont={\normalfont\sffamily\color{white}\hspace{-1em}},
	frametitlebackgroundcolor=black!50!white,
	nobreak,
}

% Define a custom environment for command-line snapshots
\newenvironment{commandline}{
	\medskip
	\begin{mdframed}[style=commandline]
}{
	\end{mdframed}
	\medskip
}

%----------------------------------------------------------------------------------------
%	FILE CONTENTS ENVIRONMENT
%----------------------------------------------------------------------------------------

% Usage:
% \begin{file}[optional filename, defaults to "File"]
%	File contents, for example, with a listings environment
% \end{file}

\mdfdefinestyle{file}{
	innertopmargin=1.6\baselineskip,
	innerbottommargin=0.8\baselineskip,
	topline=false, bottomline=false,
	leftline=false, rightline=false,
	leftmargin=2cm,
	rightmargin=2cm,
	singleextra={%
		\draw[fill=black!10!white](P)++(0,-1.2em)rectangle(P-|O);
		\node[anchor=north west]
		at(P-|O){\ttfamily\mdfilename};
		%
		\def\l{3em}
		\draw(O-|P)++(-\l,0)--++(\l,\l)--(P)--(P-|O)--(O)--cycle;
		\draw(O-|P)++(-\l,0)--++(0,\l)--++(\l,0);
	},
	nobreak,
}

% Define a custom environment for file contents
\newenvironment{file}[1][File]{ % Set the default filename to "File"
	\medskip
	\newcommand{\mdfilename}{#1}
	\begin{mdframed}[style=file]
}{
	\end{mdframed}
	\medskip
}

%----------------------------------------------------------------------------------------
%	NUMBERED QUESTIONS ENVIRONMENT
%----------------------------------------------------------------------------------------

% Usage:
% \begin{question}[optional title]
%	Question contents
% \end{question}

\mdfdefinestyle{question}{
	innertopmargin=1.2\baselineskip,
	innerbottommargin=0.8\baselineskip,
	roundcorner=5pt,
	nobreak,
	singleextra={%
		\draw(P-|O)node[xshift=1em,anchor=west,fill=white,draw,rounded corners=5pt]{%
		Question \theQuestion\questionTitle};
	},
}

\newcounter{Question} % Stores the current question number that gets iterated with each new question

% Define a custom environment for numbered questions
\newenvironment{question}[1][\unskip]{
	\bigskip
	\stepcounter{Question}
	\newcommand{\questionTitle}{~#1}
	\begin{mdframed}[style=question]
}{
	\end{mdframed}
	\medskip
}

%----------------------------------------------------------------------------------------
%	WARNING TEXT ENVIRONMENT
%----------------------------------------------------------------------------------------

% Usage:
% \begin{warn}[optional title, defaults to "Warning:"]
%	Contents
% \end{warn}

\mdfdefinestyle{warning}{
	topline=false, bottomline=false,
	leftline=false, rightline=false,
	nobreak,
	singleextra={%
		\draw(P-|O)++(-0.5em,0)node(tmp1){};
		\draw(P-|O)++(0.5em,0)node(tmp2){};
		\fill[black,rotate around={45:(P-|O)}](tmp1)rectangle(tmp2);
		\node at(P-|O){\color{white}\scriptsize\bf !};
		\draw[very thick](P-|O)++(0,-1em)--(O);%--(O-|P);
	}
}

% Define a custom environment for warning text
\newenvironment{warn}[1][Warning:]{ % Set the default warning to "Warning:"
	\medskip
	\begin{mdframed}[style=warning]
		\noindent{\textbf{#1}}
}{
	\end{mdframed}
}

%----------------------------------------------------------------------------------------
%	INFORMATION ENVIRONMENT
%----------------------------------------------------------------------------------------

% Usage:
% \begin{info}[optional title, defaults to "Info:"]
% 	contents
% 	\end{info}

\mdfdefinestyle{info}{%
	topline=false, bottomline=false,
	leftline=false, rightline=false,
	nobreak,
	singleextra={%
		\fill[black](P-|O)circle[radius=0.4em];
		\node at(P-|O){\color{white}\scriptsize\bf i};
		\draw[very thick](P-|O)++(0,-0.8em)--(O);%--(O-|P);
	}
}

% Define a custom environment for information
\newenvironment{info}[1][Info:]{ % Set the default title to "Info:"
	\medskip
	\begin{mdframed}[style=info]
		\noindent{\textbf{#1}}
}{
	\end{mdframed}
}

\title{Scene Segmentation}
\author{Alex Konrad\\ \texttt{akonrad@uci.edu}}
\date{CS216 Final Project --- \today}
\begin{document}
\maketitle
\section*{Introduction}

Hi, I’m Alex Konrad and I chose to work on scene segmentation using
Conditional Random Fields, recreating a paper in Python code. I'll give
an overview of the paper and talk about my experience in implementation.

\section{Project Overview}

Scene Segmentation is the task of partitioning an image into regions
corresponding to the objects in them and assigning the correct class
labels to a given region.

For this project I mostly aimed to reproduce a paper by Verbeek and
Triggs from 2007 that applied Conditional Random Fields to the
scene segmentation problem.

\section{Data Sets}

Following Verbeeks and Triggs I applied the classifier
to the MSRC 9-class label object recognition dataset.

\section{Algorithms} 

A conditional random field is a structured prediction classifier that
can take context into account when making predictions by using a
graphical model to add dependencies between regions.

As you remember from class, conditional random fields are a distinct
type of Markov random field because they directly model the conditional
distribution, P(Y|X), rather than the joint distribution P(X,Y). 



\subsection{Features}

To build features, I decomposed the image into 16x16 patches to build
a feature vector. For each patch, I compute the 128-dimensional
SIFT descriptor vector (using OpenCV), a 36-dimensional color hue descriptor
vector, and a position vector indicating which patch it belongs in.

But the CRF classifier doesn't work directly with the image feature data.
For each of these features, I cluster the feature response from
every patch in the entire training set together, and the actual features
are indicator vectors which designate the centroid assignment for that patch.
This allows us to classify a patch based on other most similar patches.

\section{Results}

Provide clear documentation of your results.   When possible, show visualizations of intermediate processing steps.  If you tried several different variants of an algorithm show a side-by-side comparison of how they worked on some of your data. If possible, include both qualitative visualizations of results (e.g., example outputs of an object detector on some test images) as well as quantitative evaluations (e.g., accuracy numbers, precision-recall curves etc.)

\section{Assessment and Evaluation}

Provide a discussion on your evaluation and assessment of the results:   having completed the project do you think the task was difficult or easy for a computer algorithm to try to solve? what were the limitations of your approach? what were the successes? how might you build a better system if you had more time? what aspects of the data limited your results? what was the weakest link? And so forth. Be honest in your assessment of your results. Similarly if the results are amazing, you should explain why if possible. You will earn full points if you write a well-written comprehensive report, even if your results are not very good... conversely you will get a low score if you write a poor report, even if you have very nice results.


\begin{thebibliography}{widest entry}
 \bibitem{VT} Scene Segmentation with Conditional Random Fields Learned from Partially Labeled Images. Jakob Verbeek and Bill Triggs, 2007. http://ljk.imag.fr/membres/Bill.Triggs/pubs/Verbeek-nips07.pdf
 \bibitem{CRF} An Introduction to Conditional Random
Fields for Relational Learning. Charles Sutton and Andrew McCallum, 2012. https://people.cs.umass.edu/~mccallum/papers/crf-tutorial.pdf
\bibitem{SIFT} Distinctive Image Features
from Scale-Invariant Keypoints. David G. Lowe, 2004.  https://www.cs.ubc.ca/~lowe/papers/ijcv04.pdf
\end{thebibliography}


\section*{Appendix}

% File contents
\begin{file}[hello.py]
\begin{lstlisting}[language=Python]
#! /usr/bin/python

import sys
sys.stdout.write("Hello World!\n")
\end{lstlisting}
\end{file}

\end{document}
